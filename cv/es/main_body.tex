% =====================================================
% ===================== ESPAÑOL =======================
% =====================================================

\cvheader{Juan Diego Figari}{Buenos Aires, Argentina}{juancito.diego@gmail.com}{https://github.com/Yarthax23}

\section*{Perfil Profesional}
Estudiante intermedio de Ciencias de la Computación en la UBA, con orientación a sistemas.
Experiencia en el diseño e implementación de servidores event-driven en C, APIs REST en Go,
y programas funcionales en Haskell y Prolog. Fuerte énfasis en diseño explícito de protocolos,
ownership del estado y desarrollo incremental priorizando corrección sobre optimización prematura, en entornos Linux.

\section*{Habilidades}
\begin{itemize}
  \item \textbf{Lenguajes:} C, Go, Python, Haskell, Prolog, Java (básico), JavaScript, HTML/CSS
  \item \textbf{Sistemas / Conceptos:} Sockets UNIX, fundamentos de TCP, \texttt{select()}, I/O event-driven, framing de protocolos, arquitecturas cliente-servidor, ownership explícito del estado
  \item \textbf{Herramientas / Misceláneo:} Linux, Bash, Git, Valgrind, GDB, QEMU, PostgreSQL, CURL, scripting y automatización básica, wargames de seguridad (OverTheWire)
  \item \textbf{Áreas:} Algoritmos, estructuras de datos, programación de sistemas, POO, programación funcional y lógica
\end{itemize}

\section*{Proyectos}
\noindent \textbf{Servidor de Chat Event-Driven en C} \hfill \textit{2025} \\
\href{https://github.com/Yarthax23/unix-chat-server}{GitHub Repository}
\begin{itemize}
  \item Servidor de chat multi-cliente, single-process y single-threaded, utilizando sockets UNIX (\texttt{AF\_UNIX}, \texttt{SOCK\_STREAM}) y \texttt{select()}.
  \item Protocolo en capas: framing, gramática y ejecución; ownership explícito del estado y transiciones deterministas.
  \item Diseño incremental y auditable, priorizando corrección y capacidad de depuración.
\end{itemize}

\noindent \textbf{Servidor To-Do REST en Go} \hfill \textit{2025} \\
\href{https://github.com/Yarthax23/go-todo-api}{GitHub Repository}
\begin{itemize}
  \item API REST con autenticación básica, persistencia en PostgreSQL y testing con \texttt{curl}
  \item Proyecto temprano en Go, enfocado en aprender el lenguaje, el uso de la librería estándar y la estructura de backends.
\end{itemize}

\noindent \textbf{Trabajo en Programación Funcional y Lógica} \hfill \textit{2025} \\
\href{https://github.com/juop12/Exactas-PLP-cNil}{GitHub Repository}
\begin{itemize}
  \item Ejercicios académicos en Haskell y Prolog: estructuras recursivas, patrones funcionales con tipado fuerte y resolución lógica de problemas.
  \item Introducción a convenciones de commits y prácticas básicas de TDD.
\end{itemize}

\section*{Educación}
\textbf{Licenciatura en Ciencias de la Computación} — Universidad de Buenos Aires, en curso (53\% completado)

\noindent \textbf{Bachillerato Bilingüe en Ciencias e Informática} — Liceo Franco Argentino Jean Mermoz

\section*{Idiomas}
\begin{itemize}
  \item Español: Nativo
  \item Inglés: Avanzado
  \item Francés: Avanzado
\end{itemize}

\section*{Experiencia Laboral}
\textbf{McDonald's - Arcos Dorados S.A.} \hfill 2023-2024
\begin{itemize}
  \item Atención al cliente, manejo de caja y cierre; reconocido como presentador destacado.
\end{itemize}

\noindent \textbf{Profesor de Piano (freelance)} \hfill 2021-2025
\begin{itemize}
  \item Clases particulares y gestión autónoma de alumnos.
\end{itemize}