% =====================================================
% ===================== ENGLISH =======================
% =====================================================

\portfolioheader{Juan Diego Figari}{Buenos Aires, Argentina}{juancito.diego@gmail.com}{https://github.com/Yarthax23}

\section*{Portfolio – Selected Projects}

% ---------- Project 0 ----------
\noindent \textbf{Geospatial Data Visualization — Buenos Aires Subway (Urbanly Exercise)}
\hfill \textit{2026}
\begin{itemize}
  \item Web application rendering subway lines and stations of Buenos Aires using real public datasets from the GCBA.
  \item Offline data transformation pipeline converting heterogeneous CSV and GeoJSON sources into normalized GeoJSON.
  \item Deliberate decision to avoid topological reconstruction of line geometries due to missing ordering and connectivity guarantees in the source data.
  \item Segment-based rendering approach: each line segment is preserved as an independent feature to avoid false connections.
  \item Focus on correctness, transparency of assumptions, and faithful representation of imperfect real-world data.
  \item \textbf{Repository:} \href{https://github.com/Yarthax23/urbanly-subte-map}{urbanly-subte-map}
\end{itemize}

\vspace{0.1cm}
\figureblock
  [0.8\textwidth]
  {0.45\textheight}
  {../urbanly-subte-map/assets/screenshots/map-overview.png}
  {Subway map overview.
  Lines and stations rendered using MapLibre GL JS from normalized GeoJSON sources.
  Each segment corresponds directly to original dataset geometry.}

\vspace{0.15cm}
\begin{itemize}
  \item \textbf{Key design focus:} Data pipeline separation, explicit assumptions, and correctness over visual overfitting.
\end{itemize}

\newpage


% ---------- Project 1 ----------
\textbf{Event-Driven Chat Server in C} \hfill \textit{2025}
\begin{itemize}
  \item Single-process, event-driven chat server using UNIX domain stream sockets and \texttt{select()}.
  \item Explicit protocol design with strict layered separation: input framing, command grammar, and execution.
  \item Manual client lifecycle management via a fixed-size client table (single-threaded, no \texttt{fork()}).
  \item Emphasis on correctness, reproducibility, and explicit ownership of server state over premature concurrency.
  \item \textbf{Repository:} \href{https://github.com/Yarthax23/unix-chat-server}{unix-chat-server}
\end{itemize}
  
\vspace{0.1cm}
\figureblock
  [0.8\textwidth]
  {0.45\textheight}
  {../chat-server/assets/diagrams/dataFlow.pdf}
  {Data flow and separation of responsibilities.
  Strict pipeline from \texttt{recv()} to \texttt{send()}, with clear separation between framing, grammar, and execution. The grammar validates input and never performs I/O or state mutation.}

\vspace{0.1cm}
\figureblock
  [0.7\textwidth]
  {0.4\textheight}
  {../chat-server/assets/screenshots/server-terminal.png}
  {Server – protocol execution.
  Event-driven \texttt{select()} loop processing a byte stream, emitting JOIN/LEAVE/QUIT events and broadcasting messages to clients in the corresponding room.}

\vspace{0.1cm}
\figureblock
  [0.7\textwidth]
  {0.35\textheight}
  {../chat-server/assets/screenshots/client-toad.png}
  {Spectator client (Toad).
  Protocol view from a passive client, observing server events and user messages, including a client leaving and rejoining with \mbox{late nickname assignment.}}

\vspace{0.2cm}
\begin{itemize}
  \item \textbf{Key design focus:} Ownership of server state, broadcast semantics, and grammar vs execution separation.
\end{itemize}


\newpage
% ---------- Project 2 ----------
\noindent \textbf{Go To-Do Server with PostgreSQL} \hfill \textit{2025}
\begin{itemize}
  \item REST-style API server written in Go with PostgreSQL persistence.
  \item Implemented basic CRUD endpoints and database interactions.
  \item Early Go learning project focused on understanding language fundamentals, standard library usage, and backend structure.
  \item Project predates adoption of structured commit conventions and formal testing.
  \item \textbf{Repository:} \href{https://github.com/Yarthax23/go-todo-api}{go-todo-api}
  % \item Screenshots / diagrams (optional): 
          % \includegraphics[width=0.8\textwidth]{screenshots/go-tareas.png}
\end{itemize}

% ---------- Project 3 ----------
\noindent \textbf{Functional Programming Project in Haskell} \hfill \textit{2025}
\begin{itemize}
  \item University project focused on functional programming using Haskell.
  \item Implemented core algorithms and data transformations using pure functions
        and strong type guarantees.
  \item Emphasis on correctness, immutability, and explicit reasoning about program behavior.
  \item Early exposure to structured commits and disciplined development practices.
  \item \textbf{Repository:} \href{https://github.com/juop12/Exactas-PLP-cNil/tree/main/tp1-programacion-funcional}{Haskell TP (PLP)}
\end{itemize}

\section*{How I Approach Systems Problems}
I prioritize correctness and explicit ownership of state before performance or scalability.
I prefer designs where control flow and side effects are visible, debuggable, and easy to
reason about. When building systems, I focus on clear boundaries, deterministic behavior,
and incremental complexity, delaying concurrency and optimization until semantics are
fully understood.